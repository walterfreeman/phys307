\documentclass[12ampt]{article}
\setlength\parindent{0pt}
\usepackage{fullpage}
\usepackage{amsmath}
\usepackage{tabto}
\setlength{\parskip}{4mm}
\def\LL{\left\langle}   % left angle bracket
\def\RR{\right\rangle}  % right angle bracket
\def\LP{\left(}         % left parenthesis
\def\RP{\right)}        % right parenthesis
\def\LB{\left\{}        % left curly bracket
\def\RB{\right\}}       % right curly bracket
\def\PAR#1#2{ {{\partial #1}\over{\partial #2}} }
\def\PARTWO#1#2{ {{\partial^2 #1}\over{\partial #2}^2} }
\def\PARTWOMIX#1#2#3{ {{\partial^2 #1}\over{\partial #2 \partial #3}} }
\newcommand{\BE}{\begin{displaymath}}
\newcommand{\EE}{\end{displaymath}}
\newcommand{\BNE}{\begin{equation}}
\newcommand{\ENE}{\end{equation}}
\newcommand{\BEA}{\begin{eqnarray}}
\newcommand{\EEA}{\nonumber\end{eqnarray}}
\newcommand{\EL}{\nonumber\\}
\newcommand{\la}[1]{\label{#1}}
\newcommand{\ie}{{\em i.e.\ }}
\newcommand{\eg}{{\em e.\,g.\ }}
\newcommand{\cf}{cf.\ }
\newcommand{\etc}{etc.\ }
\newcommand{\Tr}{{\rm tr}}
\newcommand{\etal}{{\it et al.}}
\newcommand{\OL}[1]{\overline{#1}\ } % overline
\newcommand{\OLL}[1]{\overline{\overline{#1}}\ } % double overline
\newcommand{\OON}{\frac{1}{N}} % "one over N"
\newcommand{\OOX}[1]{\frac{1}{#1}} % "one over X"



\begin{document}
\Large
\centerline{\sc{Notes on using arrays in Python}} 
\normalsize


In Python, you've been using three-component NumPy arrays for a while to represent vectors. However, NumPy arrays are much more general than that.

A NumPy array can have any number of dimensions: you may make 3-component arrays like you have been, but you can also make a $50\times3$ array (which you might do in order to hold 50 vectors, for the positions of your vibrating string elements), a $50\times50$ array to hold a large matrix,
or even a monstrosity like a $1920 \times 1080 \times 3 \times 600$ array (to hold the red/green/blue components of $1920\times1080$ pixels for each of 600 frames of a movie).

For this project, you will need to create some arrays that are $N+1 \times 3$ for the positions and velocities of all of the masses that comprise the string.

\subsection{Creating an array}

Before you use an array, you will need to tell Python its dimensions. (This is rather like declaring variables in C.) You may do this with the {\tt zeroes} function to create an array of your choice of size and shape:

\verb|position = zeros((50, 3))| will create an array that is $50 \times 3$ and store it as \verb|position|.

Once you've done this, you may address any element with syntax like \verb|position[30,2]|. If $x$, $y$, and $z$ are the 0, 1, and 2 elements of your vectors, then  \verb|position[30,2]| means ``the z-component of the 30th position''.

\subsection{Manipulating arrays, the traditional way}

Arrays and \verb|for| loops are great friends. Suppose you wanted to create a multiplication table. Try the following code:

\begin{verbatim}
from numpy import *

timestable = zeros((13,13))

for i in range(13):
    for j in range(13):
        timestable[i,j]=i*j

print (timestable)
\end{verbatim}

Notice how multiple indices are separated by a comma and that \verb|print| does sensible things when you ask it to print arrays.

You can thus use {\tt for} loops to iterate over your arrays.


\subsection{Array slicing}

In NumPy, you have already seen code that does mathematics with arrays. Consider the following code to simulate planetary orbits:

\begin{verbatim}
from numpy import *

G = 4*pi**2
pos = array([1.0,0.0,0.0])
vel = array([0.0,2*pi,0.0])
dt = 1e-2

while True:
    pos += vel * dt/2
    vel += -pos*G/(linalg.norm(pos) ** 3) * dt
    pos += vel * dt/2
    print ("C 1 1 0\nc3 0 0 0 0.1\nC 0.3 0.3 1\nct3 0 %e %e %e\nF\n" % (pos[0],pos[1],pos[2]))
\end{verbatim}

In the leapfrog update, note that writing \verb|pos += vel * dt/2| treats the entire \verb|pos| array (which we are using to represent a Cartesian vector) as one object, updating all three components at once. The rule is that {\it if you omit an index, Python will loop over it.} The above code is equivalent to the following:

\begin{verbatim}
from numpy import *

G = 4*pi**2
pos = array([1.0,0.0,0.0])
vel = array([0.0,2*pi,0.0])
dt = 1e-2

while True:
    for i in range(3):
        pos[i] += vel[i] * dt/2
    for i in range(3):
        vel[i] += -pos[i]*G/(linalg.norm(pos) ** 3) * dt
    for i in range(3):
        pos[i] += vel[i] * dt/2
print ("C 1 1 0\nc3 0 0 0 0.1\nC 0.3 0.3 1\nct3 0 %e %e %e\nF\n" % (pos[0],pos[1],pos[2]))
\end{verbatim}

However, if you have two-dimensional arrays, things can get a little bit more complex. Suppose that you are simulating a vibrating string with 100 points and have generated two $100 \times 3$ arrays called \verb|position| and \verb|velocity| to hold the position and velocity of the masses. The first index tells you which point you're talking about and runs from 0 to 99; the second index tells you which component (x, y, or z), and runs from 0 to 2. You can address those arrays as the following:

\begin{itemize}
	
	\item \verb|position[50,1]| \tabto{1.2in}-- means the $y$-component of mass number 50, a single number
	\item \verb|position[50]| \tabto{1.2in} -- means the vector position of mass number 50, a three-component object
    \item \verb|position| \tabto{1.2in} -- means the entire array, a $100\times3$ object
\end{itemize}

Replacing an index with a colon tells NumPy to iterate over that index. So for instance:
\begin{itemize}
	
	\item \verb|position[:,1]| \tabto{1.2in}-- means the $y$-component of all of the masses, a 50-component object
\end{itemize}


So, for instance, all of the following codes do the same thing:

First:
\begin{verbatim}
position = position + velocity      // update every component of every point at once
\end{verbatim}


\bigskip

Second:
\begin{verbatim}
for i in range(100):                        // iterate over every point...
    position[i] = position[i] + velocity[i] // ... and update all three components of its position
\end{verbatim}

\bigskip

Third:
\begin{verbatim}
for i in range(100):                                  // iterate over every point...
    for j in range(3):                                // ... and over Cartesian directions
        position[i,j] = position[i,j] + velocity[i,j] // ... and update that component of that point
\end{verbatim}

\bigskip

Fourth:
\begin{verbatim}
for i in range(3):                                // iterate over every Cartesian direction...
    position[:,i] = position[:,i] + velocity[:,i] // ... and update that direction for all 
                                                  // 100 points at once
\end{verbatim}


This behavior lets you avoid writing out explicit \verb|for| loops when writing NumPy code. For our class, if there is ever anything you'd like to do (for instance: ``add this vector to all the vectors in this array'', ``add 2 to all of the y-components of the vectors in this list'', etc.), please ask; we'll be happy to show you one-on-one how to do whatever you want.


\end{document}
