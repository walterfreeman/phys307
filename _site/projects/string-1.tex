\documentclass[12pt]{article}
\setlength\parindent{0pt}
\usepackage{amsmath}
\usepackage{fullpage}
\setlength{\parskip}{4mm}
\def\LL{\left\langle}   % left angle bracket
\def\RR{\right\rangle}  % right angle bracket
\def\LP{\left(}         % left parenthesis
\def\RP{\right)}        % right parenthesis
\def\LB{\left\{}        % left curly bracket
\def\RB{\right\}}       % right curly bracket
\def\PAR#1#2{ {{\partial #1}\over{\partial #2}} }
\def\PARTWO#1#2{ {{\partial^2 #1}\over{\partial #2}^2} }
\def\PARTWOMIX#1#2#3{ {{\partial^2 #1}\over{\partial #2 \partial #3}} }
\newcommand{\BE}{\begin{displaymath}}
\newcommand{\EE}{\end{displaymath}}
\newcommand{\BNE}{\begin{equation}}
\newcommand{\ENE}{\end{equation}}
\newcommand{\BEA}{\begin{eqnarray}}
\newcommand{\EEA}{\nonumber\end{eqnarray}}
\newcommand{\EL}{\nonumber\\}
\newcommand{\la}[1]{\label{#1}}
\newcommand{\ie}{{\em i.e.\ }}
\newcommand{\eg}{{\em e.\,g.\ }}
\newcommand{\cf}{cf.\ }
\newcommand{\etc}{etc.\ }
\newcommand{\Tr}{{\rm tr}}
\newcommand{\etal}{{\it et al.}}
\newcommand{\OL}[1]{\overline{#1}\ } % overline
\newcommand{\OLL}[1]{\overline{\overline{#1}}\ } % double overline
\newcommand{\OON}{\frac{1}{N}} % "one over N"
\newcommand{\OOX}[1]{\frac{1}{#1}} % "one over X"

\begin{document}
\Large
\centerline{\sc{Physics 307 Homework 6}}
\centerline{Due Thursday, 9 November, at 11 AM}
\normalsize

This homework assignment is the first of two parts. I encourage you to try to finish this assignment ahead of time
and begin the next one before the deadline of this one. This assignment requires you to write some code; the next assignment will involve using that code to study
some physics.

Your ``report'' for this assignment will just consist of keeping us up-to-date on your progress; next Thursday we'd like to see where you are, and will come around
and check on you.

{\bf Please start early.} 
I am giving you two weeks to do this assignment in order to accommodate anyone who may have some difficulty; the next assignment will be due after one more week, and will consist of answering some questions 
about vibrating strings using your simulation. This will be a bit more challenging to code than your previous one; please don't put it off until a few days before it's due.

In this assignment, you will model a stretched string (as in a guitar or violin) using a lattice elasticity model, in which point masses (``nodes'') are connected by Hooke's-law springs.
For reference, the force law on any given node $i$ (given in more detail in class) is:

\begin{align}
  \dot v_{x_i} &= \frac{1}{m} \left[ k \hat r_{{i,i-1}_x} (r_{i,i-1} - r_0) + k \hat r_{i,i+1} (r_{i,i+1} - r_0) \right] \\
  \dot v_{y_i} &= \frac{1}{m} \left[ k \hat r_{{i,i-1}_y} (r_{i,i-1} - r_0) + k \hat r_{i,i+1} (r_{i,i+1} - r_0) \right]
\end{align}

where

$\vec r_{i,j}$ is a vector pointing from the $i$'th node to the $j$'th one, and is thus equal to $(x_j-x_i)\hat i + (y_j-y_i)\hat j$, $k$ is the spring constant of a single spring, $r_0$ is its unstretched length, and $m$ is the mass of each node.
As before, you can use the ``unit vector trick'' to write $\hat r_{i,j}$ in terms of the dynamical variables in the problem:

\begin{align}
  \hat r_{{i,j}_x} &= \frac{x_i - x_j}{\sqrt{(x_i-x_j)^2 + (y_i-y_j)^2}}\\
  \hat r_{{i,j}_y} &= \frac{y_i - y_j}{\sqrt{(x_i-x_j)^2 + (y_i-y_j)^2}}
\end{align}


  
  A lattice simulation of a stretched string has the following parameters:
    \begin{itemize}
      \item{The number of nodes in the simulation $N+1$ (or the number of springs $N$)}
        \item{The linear mass density of the unstretched string $\mu$}
        \item{The unstretched length of the string $L_0$}
        \item{The tension $T$ applied to it to stretch it, which gives you the stretched length $L$}
        \item{The stiffness $\alpha$ (Young's modulus times cross-sectional area)}
      \end{itemize}

\begin{enumerate}

  \item{Determine values of $m$, $k$, and $r_0$ (the ``microscopic'' quantities of interest to your simulation) as a function of $N, \mu, L_0$ and $\alpha$ (the ``macroscopic'' quantities).}

  \item{Write a program that simulates the vibrations of a stretched string of uniform density and elasticity, as we will discuss in class. (Notes will be posted on the website, too.)
      You should be able to easily change (by setting variables) some crucial parameters of your simulation: $N, \mu, L_0$, and $\alpha$.

      You should use the leapfrog solver for your simulation, although it might be useful to code it using Euler-Cromer first as a learning exercise.

    }

  \item{Determine the initial conditions (all of the $x_i$'s; all the $y$-coordinates and velocities will be zero) for a stretched but unexcited string with tension $T$ applied to it. You will first need to determine the stretched length $L'$, then put your $N$
    masses at equidistant points between $x=0$ and $x=L'$. Implement these initial conditions, and verify that the string doesn't move.} 

  \item{Now stretch your string with sufficient tension to stretch it to around 125\% of its unstretched length, and excite it somehow. You can do this by applying some external force to it at some instant in time, or by choosing initial values for the $y_i$'s.
    Verify that it moves somewhat realistically.}

  \item{Play around with your initial conditions, along with $N$, the stretch amount, $\mu$, $L$, and so on. Discuss anything interesting that you find.}

  \item{Modify your program to track conservation of energy (recall that the elastic potential energy in each spring is $U=\frac{1}{2}k(r-r_0)^2$) and verify that energy is approximately conserved.}
\end{enumerate}

\end{document}
