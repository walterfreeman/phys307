\documentclass[12ampt]{article}
\setlength\parindent{0pt}
\usepackage{fullpage}
\usepackage{amsmath}
\setlength{\parskip}{4mm}
\def\LL{\left\langle}   % left angle bracket
\def\RR{\right\rangle}  % right angle bracket
\def\LP{\left(}         % left parenthesis
\def\RP{\right)}        % right parenthesis
\def\LB{\left\{}        % left curly bracket
\def\RB{\right\}}       % right curly bracket
\def\PAR#1#2{ {{\partial #1}\over{\partial #2}} }
\def\PARTWO#1#2{ {{\partial^2 #1}\over{\partial #2}^2} }
\def\PARTWOMIX#1#2#3{ {{\partial^2 #1}\over{\partial #2 \partial #3}} }
\newcommand{\BE}{\begin{displaymath}}
\newcommand{\EE}{\end{displaymath}}
\newcommand{\BNE}{\begin{equation}}
\newcommand{\ENE}{\end{equation}}
\newcommand{\BEA}{\begin{eqnarray}}
\newcommand{\EEA}{\nonumber\end{eqnarray}}
\newcommand{\EL}{\nonumber\\}
\newcommand{\la}[1]{\label{#1}}
\newcommand{\ie}{{\em i.e.\ }}
\newcommand{\eg}{{\em e.\,g.\ }}
\newcommand{\cf}{cf.\ }
\newcommand{\etc}{etc.\ }
\newcommand{\Tr}{{\rm tr}}
\newcommand{\etal}{{\it et al.}}
\newcommand{\OL}[1]{\overline{#1}\ } % overline
\newcommand{\OLL}[1]{\overline{\overline{#1}}\ } % double overline
\newcommand{\OON}{\frac{1}{N}} % "one over N"
\newcommand{\OOX}[1]{\frac{1}{#1}} % "one over X"



\begin{document}
\Large
\centerline{\sc{Quiz 2}}
\normalsize
\centerline{You need only your pencil for this quiz; no calculators or computers are needed.}

\begin{enumerate}

\item Suppose you are using the midpoint rule to calculate an integral. If using a stepsize of $h=10^{-3}$ gives an error $E=2\times10^{-4}$, what will the error be if the stepsize is decreased to $10^{-4}$?

\vspace{3in}

\item Someone gives you a program that they claim uses the trapezoid rule to compute an integral. You decide to test it on an integral whose analytic answer you know.

\bigskip

You observe that for a stepsize of $h=2 \times 10^{-2}$, the error in the integral is about $E=4 \times 10^{-3}$, and for a stepsize of $h=10^{-2}$, the error in the integral is $E=2 \times 10^{-3}$.

\bigskip

Is the program working correctly? How do you know?



\end{enumerate}

\end{document}
