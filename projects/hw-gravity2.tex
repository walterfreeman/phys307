\documentclass[12pt]{article}
\setlength\parindent{0pt}
\usepackage{amsmath}
\usepackage{fullpage}
\setlength{\parskip}{4mm}
\def\LL{\left\langle}   % left angle bracket
\def\RR{\right\rangle}  % right angle bracket
\def\LP{\left(}         % left parenthesis
\def\RP{\right)}        % right parenthesis
\def\LB{\left\{}        % left curly bracket
\def\RB{\right\}}       % right curly bracket
\def\PAR#1#2{ {{\partial #1}\over{\partial #2}} }
\def\PARTWO#1#2{ {{\partial^2 #1}\over{\partial #2}^2} }
\def\PARTWOMIX#1#2#3{ {{\partial^2 #1}\over{\partial #2 \partial #3}} }
\newcommand{\BE}{\begin{displaymath}}
\newcommand{\EE}{\end{displaymath}}
\newcommand{\BNE}{\begin{equation}}
\newcommand{\ENE}{\end{equation}}
\newcommand{\BEA}{\begin{eqnarray}}
\newcommand{\EEA}{\nonumber\end{eqnarray}}
\newcommand{\EL}{\nonumber\\}
\newcommand{\la}[1]{\label{#1}}
\newcommand{\ie}{{\em i.e.\ }}
\newcommand{\eg}{{\em e.\,g.\ }}
\newcommand{\cf}{cf.\ }
\newcommand{\etc}{etc.\ }
\newcommand{\Tr}{{\rm tr}}
\newcommand{\etal}{{\it et al.}}
\newcommand{\OL}[1]{\overline{#1}\ } % overline
\newcommand{\OLL}[1]{\overline{\overline{#1}}\ } % double overline
\newcommand{\OON}{\frac{1}{N}} % "one over N"
\newcommand{\OOX}[1]{\frac{1}{#1}} % "one over X"

\begin{document}
\Large
\centerline{\sc{Physics 307 Homework 6}}
\centerline{Due Thursday, 25 October, at 5 PM}
\normalsize

In this homework assignment, you will use your orbit program to simulate
a highly elliptical orbits, and then a binary star.

\begin{enumerate}

  \item{Consider a highly elliptical orbit such as a comet. At aphelion, Halley's comet is 35.1 AU from the Sun and has a velocity of 0.193 AU/year.
    Use your program to simulate the orbit of Halley's comet. How close does it come to the Sun at perihelion, and how fast is it traveling there? 
    
    About how many timesteps are required to simulate this orbit reasonably accurately? Compare this to the value for a circular orbit that you determined

\item{{\bf Extra credit:} Can you think of any way to mitigate this and simulate the orbit with fewer timesteps? Implement it.}

  last week. What makes cometary orbits so much harder to simulate accurately?}

\item If you're not familiar with Kepler's laws of orbital motion, look them up. Does the
behavior of your simulation reflect Kepler's second law (qualitatively; you do not need
to measure areas!)
  
\item{Now, modify your program to simulate a binary star system with two stars
of unequal mass-- that is, two objects
responding to each other's gravity. 

    Hints:

    \begin{itemize}
      \item{If you measure mass in solar masses, $G$ still has a value of $4\pi^2$.}
      \item{You will now need to use $\vec F=m\vec a$ and $\vec F=\frac{Gm_1m_2}{r_{12}^2}$
             together to find the accelerations.}
      \item{Your count of dynamical variables has now doubled to eight:
          \begin{itemize}
            \item{$x_1, y_1, v_{x_1}, v_{y_1}$ for the first star}
            \item{$x_2, y_2, v_{x_2}, v_{y_2}$ for the second star}
          \end{itemize}
       }
     \item {
          The leapfrog prescription is still the same: 
            \begin{enumerate}
              \item Evolve the position variables (all four of them!) forward by $dt/2$
              \item Evolve the velocity variables (all four of them!) forward by $dt$ (this is the hard part)
              \item Evolve the position variables forward by $dt/2$ again
            \end{enumerate}
            Your cut-and-paste skills will come in handy here, as you will be typing
            multiple copies of very similar code. But be careful of editing errors when doing this!
          }
  
      \item{The radius vector that appears in the differential equations is now the {\it separation vector} between the two stars. The origin no longer plays any special role in the dynamics.}
      \item{Since the $x$ and $y$ that appear explicitly in the differential equations come from the components of $\hat r_{12}$, they will become $x_1-x_2$, etc.}
    \end{itemize}

\item Important: if you don't want your simulation to ``drift'' out of the viewport, 
then you will want to ensure that the total momentum is zero. See the end of this document
for how to do this.

   Run your simulation, and monitor conservation of total energy to ensure that it is behaving properly. You now can't compute energy per unit mass, since you have two masses; the total energy will be

   \begin{equation}
     E = \frac{1}{2}m_1v_1^2 + \frac{1}{2}m_2v_2^2 - \frac{Gm_1m_2}{r_{12}}
   \end{equation}

\item Now, finally, modify your code to simulate the gravitational interactions of {\it 
three} bodies. This is no more complicated than two; you just have some copy-paste work
to do, since each object now feels the force from {\it two} neighbors, rather than just one.

Play around with what you can create -- make things move in three dimensions, etc. Note that if two bodies get too close together,
they will experience a very large force that is probably too big for your timestep to accurately simulate. Can you make a stable
sun-planet-moon system?

}
  \end{enumerate}

\bigskip
\bigskip

\begin{center}
\Large Avoiding solar system drift
\end{center}

\normalsize

The total momentum of a system of two stars is

$$
\vec p =m_1 \vec v_1 + m_2 \vec v_2
$$

This gives a center-of-mass velocity of 

$$\vec v_{\rm com} = \frac{m_1 \vec v_1 + m_2 \vec v_2}{m_1+m_2}$$

By calculating this value and subtracting it from each of your objects' initial velocities,
you ensure that the total center-of-mass velocity (and thus the total momentum) is zero,
and your simulation won't drift. This can be done with code like the following:

\begin{verbatim}
double vxc,vyc; //center-of-mass velocities
vxc = (m1*vx1 + m2*vx2) / (m1+m2); 
vyc = (m1*vy1 + m2*vy2) / (m1+m2); 
vx1 = vx1 - vxc;
vy1 = vy1 - vyc;
vx2 = vx2 - vxc;
vy2 = vy2 - vyc;
\end{verbatim}

\end{document}

