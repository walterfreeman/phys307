\documentclass[12pt]{article}
\setlength\parindent{0pt}
\usepackage{fullpage}
\setlength{\parskip}{4mm}
\def\LL{\left\langle}   % left angle bracket
\def\RR{\right\rangle}  % right angle bracket
\def\LP{\left(}         % left parenthesis
\def\RP{\right)}        % right parenthesis
\def\LB{\left\{}        % left curly bracket
\def\RB{\right\}}       % right curly bracket
\def\PAR#1#2{ {{\partial #1}\over{\partial #2}} }
\def\PARTWO#1#2{ {{\partial^2 #1}\over{\partial #2}^2} }
\def\PARTWOMIX#1#2#3{ {{\partial^2 #1}\over{\partial #2 \partial #3}} }
\newcommand{\BE}{\begin{displaymath}}
\newcommand{\EE}{\end{displaymath}}
\newcommand{\BNE}{\begin{equation}}
\newcommand{\ENE}{\end{equation}}
\newcommand{\BEA}{\begin{eqnarray}}
\newcommand{\EEA}{\nonumber\end{eqnarray}}
\newcommand{\EL}{\nonumber\\}
\newcommand{\la}[1]{\label{#1}}
\newcommand{\ie}{{\em i.e.\ }}
\newcommand{\eg}{{\em e.\,g.\ }}
\newcommand{\cf}{cf.\ }
\newcommand{\etc}{etc.\ }
\newcommand{\Tr}{{\rm tr}}
\newcommand{\etal}{{\it et al.}}
\newcommand{\OL}[1]{\overline{#1}\ } % overline
\newcommand{\OLL}[1]{\overline{\overline{#1}}\ } % double overline
\newcommand{\OON}{\frac{1}{N}} % "one over N"
\newcommand{\OOX}[1]{\frac{1}{#1}} % "one over X"

\begin{document}
\Large
\centerline{\sc{Physics 307 Homework 4}}
\centerline{Due Tuesday, 10 October, at 11 AM}
\normalsize


\begin{center}
\it Note: See the notes on symplectic integrators for summaries of the different solvers used in this project.
\end{center}

Now we will solve a DE that we can't do with pen and paper. The equation of motion for a pendulum is


\begin{equation}
\PARTWO{\theta}{t}=-\frac{g}{L} \sin \theta
\end{equation}

In mechanics class you solved this by taking the small-angle approximation $\sin \theta \approx \theta$; the equation then has a solution

\begin{equation}
\theta(t) = A \sin (\omega t + \phi)
\end{equation}

where $\omega=\sqrt\frac{g}{L}$ giving a period $T=2\pi\sqrt\frac{L}{g}$. This is valid only in the limit $\theta \rightarrow 0$.

This equation is very difficult to solve without making this approximation using pen and paper, but you have a computer! 

\begin{enumerate}

\item Without making the small-angle approximation, 
write a computer program that solves Newton's law (rotational form) to compute the oscillation of a swinging pendulum using the Euler-Cromer algorithm, which 
is a first-order symplectic integrator. (It is perhaps the easiest to code of all of them.
)
\item Animate your pendulum using {\tt anim}.

\item Modify your program to determine the period of the pendulum and print it out. You can determine when a period has elapsed by looking for sign changes in $\omega$.

\item Does your program give you the result you expect (\ie $T=2\pi\sqrt\frac{L}{g}$) in the limit where the initial amplitude is small?

\item Modify your program to use the leapfrog algorithm, which is a second-order symplectic integrator.

\item Suppose a pendulum clock keeps accurate time when $\theta_{\rm max}=5^\circ$. How many seconds will it gain or lose per day if it is swinging at an angle of  $\theta_{\rm max}=20^\circ$?

\item Make a plot of the fractional deviation in the period, defined as $\Delta = \frac{T-T_0}{T_0}$. Here $T_0$ is the small-angle-limiting period, $T_0=2\pi\sqrt\frac{L}{g}$. Calculate this for a range of $\theta_{\rm max}$ from $10^{-5}$ to 2 (radians). Make a log-log plot of $\Delta$ vs. $\theta_{\rm max}$. Your plot should
cover amplitudes from around $10^{-6}$ to 2 radians. Here you might see
deviations from the small-angle period from two sources: actual physical effects, and error in your numerical solution. You should take whatever steps you can to minimize
the latter (using a reasonable stepsize, using interpolation to determine the period as you did in HW3, etc.), and then correctly interpret your data in light of the fact
that you may still have errors related to your numerical solution.

Based on what you know about the power series expansion of $\sin(x)$, comment on its appearance. Is it what you expect?

\item {\bf Graduate students and ambitious undergrads:} Modify your program to keep track of the energy anomaly $E(t) - E(0)$ (the amount by which conservation of energy is violated)
as the simulation progresses. (The potential and kinetic energy are what you think they should be -- see the symplectic integrator notes.) You can print the energy anomaly while your program is animating
using the ! passthrough character; ask me if this isn't clear from the notes on animation.

Conduct both a qualitative (i.e. watch the animation) and qualitative (plot the energy anomaly) study of the performance of the Euler, Euler-Cromer, RK2, and leapfrog
integrators for the pendulum. How large of a stepsize can you trust for short runtimes (just one period)? What about long runtimes (many thousands of periods)?

Note that I am not telling you exactly what to measure, or what to plot against what.

{\it Anyone working on this project may collaborate in figuring this out!}


\end{enumerate}



\end{document}
