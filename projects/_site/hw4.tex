\documentclass[12pt]{article}
\setlength\parindent{0pt}
\usepackage{amsmath}
\usepackage{fullpage}
\usepackage{textcomp}
\setlength{\parskip}{4mm}
\def\LL{\left\langle}   % left angle bracket
\def\RR{\right\rangle}  % right angle bracket
\def\LP{\left(}         % left parenthesis
\def\RP{\right)}        % right parenthesis
\def\LB{\left\{}        % left curly bracket
\def\RB{\right\}}       % right curly bracket
\def\PAR#1#2{ {{\partial #1}\over{\partial #2}} }
\def\PARTWO#1#2{ {{\partial^2 #1}\over{\partial #2}^2} }
\def\PARTWOMIX#1#2#3{ {{\partial^2 #1}\over{\partial #2 \partial #3}} }
\newcommand{\BE}{\begin{displaymath}}
\newcommand{\EE}{\end{displaymath}}
\newcommand{\BNE}{\begin{equation}}
\newcommand{\ENE}{\end{equation}}
\newcommand{\BEA}{\begin{eqnarray}}
\newcommand{\EEA}{\nonumber\end{eqnarray}}
\newcommand{\EL}{\nonumber\\}
\newcommand{\la}[1]{\label{#1}}
\newcommand{\ie}{{\em i.e.\ }}
\newcommand{\eg}{{\em e.\,g.\ }}
\newcommand{\cf}{cf.\ }
\newcommand{\etc}{etc.\ }
\newcommand{\Tr}{{\rm tr}}
\newcommand{\etal}{{\it et al.}}
\newcommand{\OL}[1]{\overline{#1}\ } % overline
\newcommand{\OLL}[1]{\overline{\overline{#1}}\ } % double overline
\newcommand{\OON}{\frac{1}{N}} % "one over N"
\newcommand{\OOX}[1]{\frac{1}{#1}} % "one over X"
\def\LT{\textlangle{}}
\def\RT{\textrangle{}}


\begin{document}
\Large
\centerline{\sc{Physics 307 Homework 5}}
\centerline{Due Tuesday, 18 October, at 11 AM}
\normalsize

In this homework assignment, you will simulate the Earth going around the Sun,
and then generalize your work to other orbits.

To review the discussion in class: 
For an orbit in the x-y plane, you have four dynamical variables $x$, $y$,
$v_x$, and $v_y$. The four differential equations that govern them are

\begin{align}
\dot x =& v_x\\
\dot y =& v_y\\
\dot v_x =& -\frac{GMx}{r^3} \\
\dot v_y =& -\frac{GMy}{r^3} 
\end{align}

where $M$ is the mass of the Sun.

\begin{enumerate}
\item First we will pretend that the Earth is in an exactly circular orbit;
  this is approximately the case. If we measure distance in astronomical units
    and time in years,
	\begin{itemize}
  \item{what initial conditions for the Earth's position and velocity 
    vectors will give a circular orbit with the correct radius, and}
  \item{what is the numerical value of GM?}
  \end{itemize}

  (Hint: Remember your freshman physics; what is $v^2/r$? I know some of you 
   took or taught Physics 211 with me or Physics 215 with Prof. Laiho; treat it as a 215 problem! If you're not 
   sure how to go about this, discuss with other students, but tell me how this goes on your report.)
 

\item Write code that simulates the Earth's orbit around the Sun using the Euler-Cromer algorithm. Animate your simulation. Some hints for animation:

%\begin{itemize}
%\item { You can make your planet leave behind a ``trail'' so you can visualize its orbit with the {\tt ct3} animation command. Its syntax is 
%{\tt ct3 \LT trail index\RT  \LT x\RT  \LT y\RT  \LT z\RT  \LT radius\RT }, where:

%\begin{itemize}
%\item {\tt \LT trail index \RT} controls which trail you are adding to in this case you only want to leave behind one trail, so just set this to 0
%\item {\tt \LT (x,y,z) \RT} is the location to draw the object. Since you are simulating orbits in the $xy$-plane, just set $z$ to zero.
%\item {\tt \LT r \RT} is the radius of the ball to draw.
%\end{itemize}
%}

%\item Since the Sun doesn't move, you can draw it at the origin with the {\tt c3} command.
%\item If your simulation runs too fast, reduce the timestep; if it runs too slowly, you can draw only one animation frame every 10, 100, etc. timesteps.
%To do this you might write something like:

%\begin{verbatim}
%
%int steps=0, stepsperframe=10;
%...
  %
%for (loop over time...)
  %{
    %  steps++;   // note that this means the same as "steps=steps+1"
    %  if (steps % stepsperframe)
      %  {
	%    <do anim things>
	  %  }
    %  <do physics update (Euler-Cromer-Aspel, leapfrog, etc.)>
      %}
      %\end{verbatim}  
      % 

 %     \end{itemize}

      \item{Now, let's check to see if the simulation is accurate. Since you put in initial conditions that are explicitly designed to simulate a planet orbiting in a circular
	orbit with a period of one year, you can check a couple of things:

	  \begin{enumerate}
	\item Does your planet stay at a fixed distance from the origin?
	  \item Does your planet complete an orbit in one year?
	  \item Does your planet's orbit conserve energy?
	  \end{enumerate}

	Remember, you will see {\it some} deviation from these ideals because of ``finite timestep error''. That's okay, as long as those errors get small as $dt \rightarrow 0$.

	  I'll let you test these things in any way you want.

	  The total energy per unit mass (``specific orbital energy'') is given by the sum of the potential and kinetic energy

	  \begin{equation}
	E_{\rm spec}=\frac{1}{2}(v_x^2+v_y^2) - \frac{GM}{r}.
	  \end{equation}

	Even without an exact analytic solution for orbits (in general), we can look at departures from conservation of 
	  energy as another way to gauge the error in the simulation. Modify your 

	  Modify your code to display this on the screen as the animation runs. You can do this with the {\tt T} animation command. An example of an elaborate way to do this 
	  would be:

	  \begin{verbatim}
	printf("T 0.9 -0.9\nEnergy (U + V = E): %e + %e = %e\n",potential,kinetic,potential+kinetic);
	\end{verbatim}

	(Notice that this is a {\it two-line} anim command: the first line ({\tt T 0.9 -0.9}) instructs the computer to print some text at a specified point in the window,
	 in this case near the top left. The second line contains the text to be printed.) Remember, the more information you have about what your code is doing, the more
	  physics you can learn from it, and the easier it will be to fix bugs.

      }


\item{Modify your code to use the leapfrog method. Comment on any changes you see.}

\item{Modify your initial conditions to simulate a non-circular orbit, changing the initial velocity and position of the Earth. (Note that if the total energy becomes positive, the Earth will escape the Sun's gravity.) What shape do these orbits take? At what part of the planet's orbit does it move faster? Spend at 
  least half an hour playing around with different initial conditions and seeing
    what happens, and write about that in your report.}

    \item With initial conditions appropriate for an elliptical orbit, modify Newton's law of gravity a bit. Specifically, change the force law to $F_g = \frac{GMm}{r^{2 \pm \epsilon}}$, where $\epsilon$ is some small value. Simulate your orbit again, ensuring that the planet leaves trails behind it. What happens?\footnote{
      It turns out there is some history here. In regions of very strong gravity, Einstein's general relativity modifies the force from gravity in a similar way. The orbit of 
	Mercury is close enough to the Sun that GR causes its orbit to precess about one percent of one degree every century; this was known to astronomers and unexplained at the
	time of Einstein. Einstein was aware of this, and concluded his paper introducing GR by pointing out that his correction to the law of gravity explains exactly the 
	anomaly in the precession of Mercury's orbit.}



	\item{Finally, consider what happens if you move the Sun away from the origin. Suppose you put the Sun at coordinates (0.2, 0.3). Change your code to simulate orbits around a star not at the origin.}

	\end{enumerate}

	\end{document}
