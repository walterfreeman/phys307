\documentclass[12pt]{article}
\setlength\parindent{0pt}
\usepackage{fullpage}
\setlength{\parskip}{4mm}
\def\LL{\left\langle}   % left angle bracket
\def\RR{\right\rangle}  % right angle bracket
\def\LP{\left(}         % left parenthesis
\def\RP{\right)}        % right parenthesis
\def\LB{\left\{}        % left curly bracket
\def\RB{\right\}}       % right curly bracket
\def\PAR#1#2{ {{\partial #1}\over{\partial #2}} }
\def\PARTWO#1#2{ {{\partial^2 #1}\over{\partial #2}^2} }
\def\PARTWOMIX#1#2#3{ {{\partial^2 #1}\over{\partial #2 \partial #3}} }
\newcommand{\BE}{\begin{displaymath}}
\newcommand{\EE}{\end{displaymath}}
\newcommand{\BNE}{\begin{equation}}
\newcommand{\ENE}{\end{equation}}
\newcommand{\BEA}{\begin{eqnarray}}
\newcommand{\EEA}{\nonumber\end{eqnarray}}
\newcommand{\EL}{\nonumber\\}
\newcommand{\la}[1]{\label{#1}}
\newcommand{\ie}{{\em i.e.\ }}
\newcommand{\eg}{{\em e.\,g.\ }}
\newcommand{\cf}{cf.\ }
\newcommand{\etc}{etc.\ }
\newcommand{\Tr}{{\rm tr}}
\newcommand{\etal}{{\it et al.}}
\newcommand{\OL}[1]{\overline{#1}\ } % overline
\newcommand{\OLL}[1]{\overline{\overline{#1}}\ } % double overline
\newcommand{\OON}{\frac{1}{N}} % "one over N"
\newcommand{\OOX}[1]{\frac{1}{#1}} % "one over X"

\begin{document}
\Large
\centerline{\sc{One Last Project}}

\normalsize

This project is substantially easier than the last one, and should be doable in a week. If you have not yet finished the vibrating string project but are close to finishing, you're probably better off finishing it rather than starting on this one;
this project is mostly intended for people who have already finished the previous one. 



\section{Overview}

In this project you will calculate $\pi$ using as simple a method as you could imagine, taking advantage of random numbers.

If you inscribe a circle of radius 1 inside a square of side 2, the area of the circle is $\pi$ and the area of the square is 4; thus, the ratio of their areas is $\pi/4$. 

This means that if you choose a point randomly inside the square, there is a $\pi/4$ chance that it will lie within the circle -- and, if you choose many such points, the fraction that will appear inside the circle
is also $\pi/4$. Thus, given a lot of random numbers, you can compute $\pi$!

\section{Coding: random numbers}

C has a built-in random number generator called {\tt drand48()}; it returns pseudorandom numbers between 0 and 1. (To produce numbers in another range, you can do things like {\tt x=drand48()*2-1;}, which will
produce a random number between -1 and 1.)

This is a {\it pseudorandom} number generator. Since computers are deterministic, the PRNG will produce the same sequence of random numbers each time unless ``seeded'' with the function {\tt srand48()}. The same seed
produces the same sequence; this is useful if you want a predictable way to test your program. If you want a ``truly'' random sequence, you should do something like this:

\begin{verbatim}
#include <stdio.h>
#include <stdlib.h>
#include <time.h>

int main(void)
{
  srand48(clock()); // use current clock time (in microseconds) to seed drand48
  int i;
  double x;
  for (i=0; i<100; i++)
  {
    x=drand48();
    printf("Random number: %f\n",x);
  }
}
\end{verbatim}

Note that I had to include two new headers: {\tt stdlib.h}, which contains the random number functions, and {\tt time.h}, to use {\tt clock()}.

\section{The assignment}

\begin{enumerate}

\item Write a function, {\tt estimatepi(int N)}, that generates $N$ random $(x,y)$ pairs within the square (-1, -1) - (1, 1), computes the fraction of those that lie within the unit circle, and then estimates $\pi$ as four
times that fraction. (You should not need arrays or anything fancy to do this.)

\item Run this function for $N=10$ to $10^7$ and make a table of your estimates of $\pi$. Does the trend match what you expect?

\item Now, evaluate the accuracy of your predictions in a rigorous way. Taking $N=10^5$, call your function 100 times to produce 100 estimates of $\pi$. Compute the mean and standard error of the mean of those estimates (I'll discuss this 
in class; notes on how to do this are forthcoming Thursday). Quote a central value and error estimate for $\pi$. Is your result consistent with the analytical value of $\pi$?

\item Ambitious students and grad students: Starting with the definition of the variance $\sigma^2 = \LL x^2 \RR - \LL x \RR^2$ in a single random number (which you can compute analytically), calculate the expected standard 
deviation of your $N=10^5$ estimates of $\pi$. Do they match the observed standard deviation when you actually did the 100 estimates in the previous part? Recall the difference between standard deviation and standard error.
\end{enumerate}


%
%\section{Statistics}
%
%You can use this program to compute an estimate for $\pi$, but how accurate is it? To determine this without knowing the value of $\pi$ {\it a priori}, you can repeat the calculation many times to get many estimates for $\pi$,
%then see how much those estimates fluctuate around their average.
%
%Let me define a few statistical quantities here along the way toward a computation of ``how much estimates fluctuate around the average''. In the discussion that follows, $\overline x$ means the average of $x$, and 
%$\LL something\RR$ means the average of $something$.
%
%\subsection{The variance}
%
%{\bf Variance} means ``the average of the squared difference from the mean'', and is written $\sigma^2$. The variance of many numbers $x$ is $$\sigma^2(x) = \LL (x - \overline x)^2 \RR = \LL x^2 \RR - 2\LL x\overline x\RR + \overline x^2 = \LL x^2 \RR - \LL x \RR^2.$$
%
%(This last formula is the useful one.)
%
%Note that we can use the idea of variance in two ways. We can use the variance to describe how much a collection of numbers fluctuates, but -- if we have a certain number drawn from that collection -- we can use the variance
%to characterize how much that number will likely change if we choose another one. This last is going to be the key in the development of statistics as a way to measure uncertainty.
%
%For instance, let's compute the variance of the roll of a four-sided die. The average squared value is $(1^2 + 2^2 + 3^2 + 4^2)/4 = 7.5$; the average value is 2.5. This means that the variance is (average squared value) - (average value squared = $7.5 - 2.5^2 = 1.25$.
%
%
%The variance is useful because of a simple property: the variance of the sum of things is the sum of their individual variances. For instance, if I wanted to know the variance of the sum of {\it ten} four-sided dice, I can say
%right away that it is 12.5 -- ten times the variance of a single die.
%
%\subsection{The standard deviation}
%
%The standard deviation, represented by the letter $\sigma$, of a collection of values is just the square root of the variance. The standard deviation is, formally, ``the square root of the average squared difference from
%the mean''; informally, it gives you an average deviation of each value from their average.
%
%For our four-sided dice, 
%
\end{document}
